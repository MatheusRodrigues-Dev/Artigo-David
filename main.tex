\documentclass[english]{sbc2025}%

\usepackage[misc,geometry]{ifsym} 

\usepackage{aas_macros}
\usepackage[bottom]{footmisc}
\usepackage{tabularray}
\usepackage{afterpage}
\usepackage{url}
\usepackage{pifont}
\usepackage{balance}

\setcitestyle{square}

\definecolor{engtitle}{rgb}{0.5,0.5,0.5}
\definecolor{orcidlogo}{rgb}{0.37,0.48,0.13}
\definecolor{unilogo}{rgb}{0.16, 0.26, 0.58}
\definecolor{maillogo}{rgb}{0.58, 0.16, 0.26}
\definecolor{darkblue}{rgb}{0.0,0.0,0.0}
\hypersetup{colorlinks,breaklinks,
            linkcolor=darkblue,urlcolor=darkblue,
            anchorcolor=darkblue,citecolor=darkblue}
%\hypersetup{colorlinks,citecolor=blue,linkcolor=blue,urlcolor=blue}

%%%%%%% IMPORTANT: We disable hyperlinks by default with this line, to avoid the error "\pdfendlink ended up in different nesting level" while writing.
%\hypersetup{draft}

\jid{JIS}
\issn{2763-7719}
\jtitle{Journal on Interactive Systems, 2026, 17:1}
\doi{10.5753/jis.2026.XXXX}
\copyrightstatement{This work is licensed under a Creative Commons Attribution 4.0 International License}
\jyear{2026}

\category{Research Paper}

\title[Template JIS 2026]{AVALIAÇÃO DE INTERFACES PARA APLICATIVOS DE CONTROLE DE DISPOSITIVOS IOT DE AUTOMAÇÃO}
%

\author[David et al. 2026]{

%THE ORCID IS MANDATORY FOR EACH AUTHOR 
%Institutions' names must be informed as registered in ROR (https://ror.org/), in English. 
%Keep each author's information and institution in a single line

\affil{\textbf{David Nakagawa}~\orcidlink{0000-0002-0339-6624}~\textcolor{blue}{\faEnvelopeO}~~[~{University State of Maringá}~|\href{mailto:exemplo}{~{\textit{exemplo}}}~]}
%Search for the institution in ROR and use its English name: https://ror.org/04bqqa360

\affil{\textbf{Matheus Rodrigues}~\orcidlink{0009-0007-3403-1525}~~[~{University State of Maringá}~|~\href{mailto:rodriguesmatheus309@gmail.com}{{\textit{rodriguesmatheus309@gmail.com}}}~]}
%https://ror.org/04bqqa360

\affil{\textbf{Sandro Lautenschlager}~\orcidlink{0000-0003-3219-2257}~~[~{University State of Maringá}~|~\href{mailto:srlager@uem.br}{{\textit{srlager@uem.br}}}~]}
%https://ror.org/04bqqa360

\affil{\textbf{Gislaine Camila Lapasini Leal}~\orcidlink{0000-0001-8599-0776}~~[~{University State of Maringá~}|~\href{mailto:gclleal@uem.br}{{\textit{gclleal@uem.br}}}~]}
%https://ror.org/04bqqa360
}


\begin{document}

\begin{frontmatter}

\maketitle

\begin{mail}
	Programa de Pós Graduação em Engenharia Urbana, Universidade Estadual de Maringá, Av. Colombo, 5790 - Zona 7, Maringá - PR, 87020-900 Brazil.
\end{mail}

\begin{abstract-en}
This text, formatted as a scientific article, aims to present the new SBC paper template, describing its main features and explaining how it should be used. The abstract must have between 400 and 750 words. \textit{Background:} JIS strongly encourages you to use the \href{https://www.nlm.nih.gov/bsd/policy/structured\_abstracts.html}{Structured Abstract template}. \textit{Purpose:} Structured abstracts have several advantages for authors and readers. \textit{Methods:} Use distinct, labeled sections  covering the main IMRAD elements (Introduction, Methods, Results, and Discussion) -- see \citep{sollaci2004introduction}. \textit{Results:} Structured abstracts guide authors in summarizing the content of their manuscripts precisely, facilitate the peer-review process, and enhance literature searching. \textit{Conclusion:} Get inspired by this structure and develop a structured abstract that adds value to your paper.
\end{abstract-en}

\begin{keywords}
	Proceedings, Template, SBC OpenLib, Indexing %keep keywords in a single line
\end{keywords}

% \begin{dates}
% % This information will be inserted by the editors during the production stage. Do not remove this section, and provide access to your paper latex project when sending the final version of your paper. 
% \noindent{\sffamily\textbf{Edited by:}}
% Full Name~\orcidlink{0000-0000-0000-0000}
% ~~$\mid$~~ 
% {\sffamily\textbf{Received:}} DD Month YYYY
% ~~$\bullet$~~
% {\sffamily\textbf{Accepted:}} DD Month YYYY
% ~~$\bullet$~~
% {\sffamily\textbf{Published:}} DD Month YYYY
% \end{dates}

\end{frontmatter}

\section{Introduction}
\label{sec:intro}
O conceito de cidades inteligentes consolidou-se a partir da década de 1990 como uma resposta aos desafios urbanos contemporâneos, impulsionado por tecnologias digitais como a Internet das Coisas (IoT) e sistemas de automação. Essas tecnologias têm potencial para aprimorar a qualidade de vida, otimizar serviços urbanos e promover sustentabilidade. Entretanto, seus benefícios dependem não apenas da robustez técnica das soluções, mas também da qualidade das interfaces que conectam cidadãos, gestores e dispositivos. Interfaces mal projetadas podem comprometer a adoção, a compreensão e o uso adequado dos sistemas, reduzindo significativamente seu impacto (\cite{Joao2019, Silva2022}).

Nesse contexto, o design de interface (UI) e a experiência do usuário (UX) tornam-se componentes centrais em sistemas baseados em IoT. Uma interface bem projetada deve antecipar necessidades, facilitar a navegação e garantir clareza na apresentação das funcionalidades. \cite{nielsen2007usabilidade} destacam que simplicidade, previsibilidade e consistência são pilares fundamentais para boas interfaces, enquanto \cite{Norman2002} aponta que produtos capazes de gerar prazer e satisfação promovem maior engajamento e aceitação tecnológica.

O design de interfaces representa, portanto, a principal ponte entre usuário e sistema. É por meio dele que informações são compreendidas, comandos são acionados e dispositivos são controlados. Em soluções de iluminação inteligente, por exemplo, mesmo que sensores e atuadores funcionem corretamente, uma interface complexa pode impedir configurações adequadas, reduzindo a eficácia da tecnologia (\cite{Norman2002}).

Além disso, a crescente portabilidade dos aplicativos, acessíveis via computadores, smartphones e tablets, amplia o potencial da IoT na vida urbana, mas também impõe desafios adicionais ao design. A diversidade de dispositivos exige interfaces responsivas, consistentes e adaptadas ao contexto de uso.

A avaliação de usabilidade desempenha papel essencial nesse processo ao identificar dificuldades, validar decisões de design e orientar melhorias. Conforme \cite{AlmeidaAbreu2018}, testes de usabilidade evitam desperdícios de tempo e recursos, enquanto \cite{Santos2019} destacam que permitem conhecer melhor os usuários e detectar oportunidades de aprimoramento. Importante notar que a aplicação não precisa estar totalmente funcional: análises de tarefas, protótipos de baixa fidelidade e avaliações heurísticas podem ser conduzidas conforme os objetivos da pesquisa.

\cite{Nielsen1994} propõe dez princípios heurísticos amplamente adotados para avaliar interfaces. Esses princípios incluem visibilidade do status do sistema, correspondência com o mundo real, consistência e padrões, prevenção de erros, reconhecimento em vez de memorização, eficiência, design minimalista, diagnóstico e recuperação de erros, ajuda e documentação. A técnica dispensa participação direta de usuários e baseia-se na inspeção sistemática de especialistas capazes de identificar desconformidades estruturais, funcionais e de interação (\cite{santa2020avaliaccao}).

Esses princípios possibilitam ao avaliador identificar problemas que afetam a experiência do usuário, como inconsistências visuais, excesso de etapas, vocabulário inadequado ou ausência de feedback. A aplicação sistemática das heurísticas fornece um panorama confiável sobre a qualidade do design antes mesmo de testes com usuários.

Diante desse cenário, esta pesquisa busca avaliar a usabilidade de três aplicativos de controle de dispositivos IoT desenvolvidos pela Smart Sensor Design. A avaliação heurística, baseada nos princípios de \cite{Nielsen1994}, teve como objetivo identificar problemas de usabilidade e propor melhorias. Essa abordagem integrada permite analisar a conformidade das interfaces, sua eficácia e a experiência de uso, além de levantar hipóteses sobre falhas de concepção que possam comprometer o desempenho das aplicações.

Aplicativos de automação precisam oferecer experiências positivas, claras e intuitivas para alcançar seu propósito. Experiências negativas geram resistência, diminuem o engajamento e prejudicam a disseminação tecnológica, um desafio crítico para a consolidação de cidades inteligentes. Avaliar interfaces em contexto real, como neste estudo de caso, permite examinar a qualidade do design, sua adequação ao público-alvo e o papel do fator humano no desenvolvimento.

Em síntese, esta pesquisa demonstra que interfaces bem projetadas, intuitivas e alinhadas às necessidades dos usuários têm maior probabilidade de serem aceitas e adotadas. Ao evidenciar como o design impacta a experiência do usuário e a adoção de tecnologias IoT, este estudo contribui para o avanço de soluções urbanas mais eficientes, acessíveis e coerentes com os princípios das cidades inteligentes.

\section{metodológica}
A pesquisa foi conduzida por meio de um estudo de caso aplicado a três aplicativos, desenvolvidos pela Smart Sensor Design (SSD), startup especializada em soluções tecnológicas voltadas ao saneamento, automação e Internet das Coisas (IoT). O presente estudo busca compreender como usuários e especialistas interagem com as interfaces dos aplicativos avaliados, identificando dificuldades, padrões de uso e oportunidades de melhoria.

Para a avaliação heurística dessa pesquisa, serão utilizados os seguintes aplicativos em desenvolvimento pela equipe da Smart Sensor Design:

\begin{itemize}
	\item \textbf{Smart Energy Meter:} Este aplicativo tem como objetivo criar uma interface de sensores de controle de consumo de energia de eletrodomésticos e aparelhos eletroeletrônicos. Ele pretende permitir que o usuário monitore e gerencie o consumo de energia em tempo real, ajude na tomada de decisões, perceba consumo excessivo e identifique a necessidade de ajustes para otimizar o consumo energético. Utilizando sensores indutivos de última geração conectados a um microcontrolador ESP32, os dados coletados são enviados para um servidor, permitindo monitoramento conveniente e detalhado;
	\item \textbf{Smart Fish Finder:} Este aplicativo é voltado para a interação com sensores de pesca. O Smart Fish Finder pretende monitorar a atividade de pesca em tempo real, incluindo a localização de cardumes, profundidade da água e temperatura, proporcionando uma experiência de pesca otimizada através de uma interface de fácil uso. Ele ajuda os pescadores a tomarem decisões informadas sobre onde e quando pescar, aumentando a eficiência e o sucesso da pesca;
	\item \textbf{Smart Flow Hall:} Este aplicativo visa controlar sensores de fluxo de água utilizando o princípio do efeito Hall. Ele tem como objetivo proporcionar precisão excepcional e longevidade no monitoramento do consumo de água, ajudando na tomada de decisões e na otimização do uso dos recursos hídricos. A tecnologia minimiza componentes mecânicos, resultando em menor necessidade de manutenção e maior confiabilidade.
\end{itemize}

Para isso, foi realizada a avaliação heurística, método amplamente utilizado na análise de usabilidade por meio dos dez princípios de Jakob Nielsen. A avaliação utilizou um checklist semiestruturado, elaborado com base nos dez princípios heurísticos de \cite{Nielsen1994}.

O processo de avaliação foi conduzido em quatro rodadas complementares, cada uma com objetivos e níveis de maturidade distintos, conforme ilustrado na Figura \ref{MetodoloEtapas}. 

\begin{figure}[!ht]
    \begin{center}
        \includegraphics[width=\columnwidth]{figures/metodologia/etapas.png}
        \caption{Etapas metodológicas da avaliação heurística}
        \label{MetodoloEtapas}
    \end{center}
\end{figure}

A primeira rodada, realizada pelo pesquisador, envolveu a análise inicial de protótipos desenvolvidos em Figma e versões beta distribuídas via TestFlight, permitindo identificar problemas estruturais de navegação, layout e organização das informações. Na segunda rodada, o pesquisador avaliou versões experimentais disponibilizadas por links restritos nas lojas Google Play Store e Apple App Store, o que possibilitou testar os aplicativos em condições mais próximas do uso real e observar questões funcionais emergentes. A terceira rodada consistiu em uma revisão sistemática das melhorias implementadas, utilizando novamente os protótipos e as versões instaladas nos dispositivos, com o objetivo de verificar a correção dos problemas previamente identificados. 

Por fim, a quarta rodada foi conduzida por dois especialistas externos, que aplicaram um checklist semiestruturado baseado nas heurísticas de Nielsen, validando os achados das fases anteriores e oferecendo uma análise técnica independente das versões mais recentes dos aplicativos.

Essa estrutura metodológica possibilitou combinar a familiaridade técnica do pesquisador com a visão crítica externa dos especialistas, oferecendo uma análise robusta sobre a usabilidade dos aplicativos desenvolvidos.

Os registros foram posteriormente analisados de forma sistemática, possibilitando comparar apps, agrupar problemas por categorias heurísticas e estabelecer recomendações específicas para cada interface.

Por envolver a participação de especialistas externos, esta etapa foi submetida à avaliação do Comitê Permanente de Ética em Pesquisa com Seres Humanos (COPEP) da Universidade Estadual de Maringá. O estudo foi aprovado sob o número CAAE 73814323.6.0000.0104, Parecer 6.336.076, conforme as diretrizes da Resolução CNS/MS nº 466/2012.

A inclusão desses três aplicativos em desenvolvimento (Smart Energy Meter, Smart Fish Finder e Smart Flow Hall) visa avaliar a eficiência de suas interfaces e sua adesão pelos usuários, contribuindo para o aprimoramento das soluções tecnológicas oferecidas pela Smart Sensor Design. 

\subsection{Instrumentos Utilizados}

Cada avaliador analisou as interfaces com base nesses critérios, registrando problemas encontrados e atribuindo níveis de gravidade de acordo com o impacto, frequência e persistência das falhas.

Da avaliação heurística, foram analisados os resultados dos checklists preenchidos pelo próprio pesquisador. Primeiramente, o pesquisador tabulou as respostas dos pontos de resposta objetiva para análise quantitativa do que foi verificado nas respostas em 10 categorias de dados.

Foi organizada uma tabela para exposição dos resultados dos três aplicativos em desenvolvimento pela Smart Sensor Design: Smart Energy Meter, Smart Fish Finder e Smart Flow Hall. Na tabela \ref{tabela-comparacao}, estão apresentados os pontos considerados na organização dos resultados com base em \cite{Nielsen1994}.


\begin{table*}[!ht]
    \caption{Categorias e critérios avaliados nos aplicativos analisados.}
    \label{tabela-comparacao}
    \centering
    \begin{tabular}{ll}
        \hline
        \textbf{Categoria} & \textbf{Critérios de verificação} \\ \hline

        & \\
        \textbf{1. Diálogos simples e naturais} & A interface comunica de forma clara o que o usuário deve fazer? \\
                                                & As informações estão bem agrupadas e são fáceis de localizar? \\
                                                & A sequência de passos para concluir tarefas é simples e intuitiva? \\
                & \\
        \textbf{2. Linguagem do usuário} &
        O vocabulário utilizado é adequado ao perfil do usuário? \\
            & \\
        \textbf{3. Sobrecarga de memória} &
        O usuário consegue operar sem lembrar comandos específicos? \\
        & Há muitos comandos particulares que exigem aprendizado? \\
        & Existem comandos genéricos que evitam sobrecarga cognitiva? \\
            & \\
        \textbf{4. Consistência} &
        Os elementos e operações aparecem sempre no mesmo local? \\
        & A consistência melhora a previsibilidade da interação? \\
        & A disposição uniforme aumenta confiança na exploração da interface? \\
            & \\
        \textbf{5. Feedback} &
        A interface informa continuamente o estado da tarefa? \\
        & O tempo de resposta é adequado para indicar reação imediata? \\
            & \\
        \textbf{6. Saídas claramente marcadas} &
        As saídas e cancelamentos são facilmente identificáveis? \\
        & O sistema fornece formas rápidas de cancelar, desfazer ou sair? \\
            & \\
        \textbf{7. Atalhos} &
        Existem atalhos acessíveis para usuários experientes? \\
        & Esses atalhos são simples e fáceis de aprender? \\
            & \\
        \textbf{8. Mensagens de erro} &
        As mensagens são claras e indicam o que ocorreu? \\
        & Elas ajudam o usuário a corrigir o problema? \\
            & \\
        \textbf{9. Ajuda} &
        Existe algum mecanismo de ajuda disponível? \\
        & A ausência de ajuda se justifica pela simplicidade da interface? \\
        & Seria necessário algum recurso adicional de ajuda? \\
            & \\
        \textbf{10. Documentação} &
        Há documentação acessível ao usuário? \\
        & A ausência de documentação é justificável? \\
        & Seria necessária documentação complementar? \\
            & \\
        \hline
    \end{tabular}
\end{table*}

Essa abordagem permitiu uma compreensão detalhada das interfaces dos aplicativos em desenvolvimento, destacando áreas de melhoria e garantindo que as interfaces sejam intuitivas e eficazes para os usuários finais.

\section{Resultados e Discussão}
Os resultados apresentados nesta seção refletem o estágio atual do desenvolvimento, e serão complementados em análises posteriores, acompanhando a evolução dos aplicativos.

\subsection{Resultados da Avaliação Heurística Smart Energy Meter}
A primeira etapa da avaliação concentrou-se na análise do estado inicial do Smart Energy Meter, permitindo identificar as principais limitações de usabilidade antes da implementação de melhorias. O aplicativo apresentava um conjunto robusto de funcionalidades, incluindo gráficos de consumo, tabelas informativas, relatórios e uma seção de ajuda relativamente estruturada. No entanto, a forma como essas informações eram organizadas e apresentadas impunha uma carga cognitiva elevada, especialmente para usuários sem experiência prévia em monitoramento energético, um problema clássico discutido por \cite{Nielsen1994} ao abordar a necessidade de simplicidade e clareza nas interfaces.

A interface utilizava terminologia tecnicamente correta, mas pouco acessível ao público geral. Termos como “corrente máxima”, “demanda contratada” e “kWh acumulado” apareciam sem explicações contextuais, exigindo conhecimento especializado para sua interpretação. Embora os gráficos estivessem visualmente bem construídos, sua compreensão dependia de familiaridade prévia com conceitos elétricos, e não havia legendas, dicas visuais ou orientações que auxiliassem o usuário a interpretar os dados apresentados.

O sistema de feedback inicial era funcional, mas limitado. Mensagens exibidas após falhas de login, erros de cadastro ou operações incorretas eram genéricas e pouco informativas, o que dificultava a identificação precisa do problema. Além disso, ações que exigiam tempo de processamento, como atualização de sensores ou carregamento de dados, não apresentavam indicadores de progresso. A ausência desse tipo de feedback visual poderia gerar a percepção de inatividade ou falha do sistema, contrariando recomendações fundamentais de visibilidade do status do sistema presentes na literatura de usabilidade.

A avaliação também identificou fragilidades na prevenção de erros. O aplicativo permitia que o usuário removesse dispositivos ou redefinisse configurações sem qualquer diálogo de confirmação, o que poderia resultar em perdas irreversíveis. Esse aspecto demonstrou falta de mecanismos de segurança para ações críticas, um requisito essencial em interfaces que lidam com dados sensíveis e controle de dispositivos.

Outro conjunto de limitações esteve relacionado à estrutura visual e à orientação à navegação. Apesar da interface apresentar poucos elementos supérfluos, característica alinhada ao design minimalista, faltavam marcadores visuais, ícones explicativos e distinções claras entre elementos interativos. Assim, embora a estrutura fosse objetivar, ela não fornecia suporte adequado para orientar o usuário na interação com o sistema. Problemas adicionais incluíam baixa saliência de botões, ausência de um fluxo inicial para orientar o primeiro uso e inexistência de mensagens que indicassem a necessidade de conectar um sensor para visualizar dados.

Em síntese, a primeira etapa revelou que o Smart Energy Meter apresentava uma base funcional sólida, porém com limitações significativas em múltiplos princípios heurísticos. Os problemas identificados concentraram-se sobretudo na visibilidade do estado do sistema, severamente limitada pela ausência de indicadores de progresso, e na correspondência com o mundo real, devido ao uso recorrente de terminologia técnica sem explicações adequadas, como “kWh acumulado” e “demanda contratada”. Falhas no controle e liberdade do usuário também foram classificadas como críticas, já que ações de alto impacto, como a remoção de dispositivos, podiam ser executadas sem confirmações.

Princípio de Nielsen Estado na Rodada 1 Gravidade Evidência Observada Visibilidade do Estado Severamente limitada Alta Ausência de indicadores de progresso Correspondência com Mundo Real Termos técnicos sem explicação Alta "kWh acumulado", "demanda contratada" Controle do Usuário Ações irreversíveis sem confirmação Crítica Remoção de dispositivos sem validação Consistência Parcialmente mantida Média Padrões visuais básicos presentes Prevenção de Erros Mecanismos insuficientes Alta Validação de formulários ausente Reconhecimento vs Memorização Exigia memorização excessiva Alta Funcionalidades pouco intuitivas Flexibilidade Navegação linear rígida Média Sequência fixa de passos Estética Minimalista Aplicada de forma prejudicial Média Falta de elementos orientadores Recuperação de Erros Mensagens genéricas Alta Feedback não contextualizado Documentação Técnica e complexa Média Seção de ajuda pouco acessível

O diagnóstico inicial evidenciou lacunas estruturais que impactavam diretamente a compreensibilidade, previsibilidade e segurança de uso do aplicativo. Esses resultados forneceram insumos essenciais para orientar as intervenções nas etapas subsequentes, permitindo que os ajustes se concentrassem nos aspectos de maior gravidade e nos princípios heurísticos mais comprometidos. Frase para conectar da Primeira a Segunda e terceira Etapa A segunda etapa da avaliação heurística do Smart Energy Meter destacou avanços importantes em relação aos problemas identificados na rodada inicial, mas também expôs limitações que ainda comprometiam a interpretação e comportamento do aplicativo. Embora melhorias preliminares tenham sido implementadas, como tooltips, reorganização de certos elementos visuais e inclusão de ícones explicativos, os resultados indicaram que a interface permanecia distante de fornecer um suporte interpretativo adequado ao usuário.

Com a mudança observou-se que o aplicativo passou a informar o andamento de algumas operações, mas ainda não oferecia indicadores capazes de contextualizar os dados. Conforme destacado por \cite{Shneiderman2016}, sistemas orientados ao usuário devem não apenas exibir informações, mas também ajudar a interpretá-las. Contudo, o Smart Energy Meter ainda exibia valores absolutos de consumo sem indicar se representavam um padrão baixo, moderado ou elevado, o que mantinha alta a carga cognitiva. A inclusão inicial de cores indicativas (verde, amarelo e vermelho) representou um progresso, mas o aplicativo continuava sem estabelecer relações diretas com o cotidiano. A ausência de analogias práticas, como equivalentes em eletrodomésticos ou tempo de uso, restringia a compreensão dos usuários leigos, conforme também observado na literatura (\cite{krug2008não}).

A segunda etapa da avaliação heurística do Smart Energy Meter destacou avanços importantes em relação aos problemas identificados na rodada inicial, mas também expôs limitações que ainda comprometiam a interpretação e comportamento do aplicativo. Embora melhorias preliminares tenham sido implementadas, como tooltips, reorganização de certos elementos visuais e inclusão de ícones explicativos, os resultados indicaram que a interface permanecia distante de fornecer um suporte interpretativo adequado ao usuário.

Com a mudança observou-se que o aplicativo passou a informar o andamento de algumas operações, mas ainda não oferecia indicadores capazes de contextualizar os dados. Conforme destacado por \cite{Shneiderman2016}, sistemas orientados ao usuário devem não apenas exibir informações, mas também ajudar a interpretá-las. Contudo, o Smart Energy Meter ainda exibia valores absolutos de consumo sem indicar se representavam um padrão baixo, moderado ou elevado, o que mantinha alta a carga cognitiva.

A inclusão inicial de cores indicativas (verde, amarelo e vermelho) representou um progresso, mas o aplicativo continuava sem estabelecer relações diretas com o cotidiano. A ausência de analogias práticas, como equivalentes em eletrodomésticos ou tempo de uso, restringia a compreensão dos usuários leigos, conforme também observado na literatura (\cite{krug2008não}).

A etapa 2 também destacou fragilidade, uma vez que operações de remoção de dispositivos e exclusão de históricos continuavam sem mensagens de confirmação robustas. Embora alguns alertas tenham sido parcialmente introduzidos, eles eram ainda sucintos e pouco informativos quanto às consequências da ação.

Verificou-se a evolução moderada. A estrutura dos gráficos e principais indicadores começou a ser reorganizada, dando maior destaque ao consumo mensal e às variações percentuais. No entanto, persistiam inconsistências no uso da paleta de cores e na diferenciação entre botões primários e secundários, algo também destacado por avaliações anteriores.

A etapa também mostrou avanços relativos à prevenção de erros, sobretudo com melhorias nas validações básicas de formulários. Entretanto, a ausência de mensagens preventivas em ações críticas continuava evidente, o que limitava a segurança da navegação.
Quanto a redução parcial da carga cognitiva: tooltips e ícones auxiliaram na compreensão de termos complexos, porém não eliminaram a necessidade de interpretação mental dos dados técnicos.

Em termos de eficiência e flexibilidade de uso, pequenas melhorias foram registradas, como a reorganização de menus e a priorização de indicadores essenciais. Ainda assim, a navegação continuava predominantemente linear, com poucos atalhos ou alternativas para realizar tarefas de forma mais rápida.

Finalmente, verificou-se que a documentação e o suporte passaram por revisão textual, reduzindo erros e melhorando a clareza. Contudo, ainda permaneciam inacessíveis para muitos usuários por não estarem integrados à interação e por não oferecerem exemplos práticos.

A terceira etapa da avaliação heurística evidenciou uma evolução substancial no Smart Energy Meter, marcada por intervenções diretamente orientadas pelos achados das duas primeiras rodadas. Diferentemente da etapa anterior, cujas melhorias foram principalmente incrementais, a terceira rodada resultou em transformações estruturais perceptíveis na apresentação das informações, no feedback do sistema e na orientação ao usuário.

Como ilustrado na Figura \ref{Fig1}, a implementação de barras de progresso, indicadores de atualização em tempo real e mensagens explicativas representou um avanço significativo em termos de visibilidade do estado do sistema. Esses elementos reduziram a sensação de inércia relatada nas etapas anteriores e se alinham com as recomendações de \cite{Norman2002} sobre previsibilidade e transparência. Além disso, o uso de classificações interpretativas de consumo (baixo, moderado, elevado) contribuiu para diminuir a carga cognitiva e tornar a leitura mais acessível a usuários iniciantes.

\begin{figure}[!ht]
    \begin{center}
        \includegraphics[width=\columnwidth]{figures/resultado/Imagem1.png}
        \caption{Evolução da visibilidade do sistema – Smart Energy Meter}
        \label{Fig1}
    \end{center}
\end{figure}

A terceira rodada também consolidou melhorias na tradução de métricas técnicas para referências cotidianas. Embora analogias completas ainda não tivessem sido implementadas, as interfaces passaram a recorrer a cores, descrições simplificadas e elementos visuais mais intuitivos — mudanças visíveis na Figura \ref{Fig2}. A sugestão de comparações práticas, como equivalências baseadas no uso de eletrodomésticos, emergiu como recomendação importante, reforçando a necessidade de aproximar o modelo conceitual do sistema ao modelo mental do usuário (\cite{krug2008não, Nielsen1994}).

No âmbito do controle do usuário, a inclusão de mensagens de confirmação antes de ações críticas mitigou parte dos riscos associados à irreversibilidade das interações. Apesar disso, a ausência de uma função de desfazer permaneceu como limitação relevante, indicando que melhorias adicionais ainda são necessárias nesse aspecto.

\begin{figure}[!ht]
    \begin{center}
        \includegraphics[width=\columnwidth]{figures/resultado/Imagem2.png}
        \caption{Evolução da Consistência e Padrões – Smart Energy Meter}
        \label{Fig2}
    \end{center}
\end{figure}

Também se observou um amadurecimento visual e funcional quanto à consistência e aos padrões da interface. As cores passaram a seguir uma lógica uniforme, os ícones foram parcialmente padronizados e a organização dos menus tornou-se mais intuitiva. Embora ainda houvesse inconsistências, como a diferenciação entre botões primários e secundários, a evolução em relação às rodadas anteriores foi expressiva.

No que se refere à prevenção e recuperação de erros, a terceira rodada trouxe avanços significativos com a introdução de validação em tempo real, mensagens de erro mais específicas e orientações contextualizadas para correções, conforme pode ser observado na Figura \ref{Fig2}. Essas melhorias evidenciam maior alinhamento com as recomendações de \cite{Dix2004}, especialmente no fornecimento de pistas adequadas durante o preenchimento de formulários.

A criação de atalhos rápidos entre gráficos, botões de navegação ("voltar" e "próximo") e a reorganização dos menus, mudanças ilustradas na Figura \ref{Fig3}, contribuíram para reduzir a rigidez da navegação linear observada previamente, aproximando o sistema de um fluxo mais eficiente e flexível.

\begin{figure}[!ht]
    \begin{center}
        \includegraphics[width=\columnwidth]{figures/resultado/Imagem3.png}
        \caption{Evolução da Ajuda e Documentação – Smart Energy Meter}
        \label{Fig3}
    \end{center}
\end{figure}

Por fim, mesmo sem um tutorial completo ou um processo de onboarding estruturado, a adição de pequenas mensagens de orientação contextual representou um avanço relevante, reduzindo a dependência de documentação externa e facilitando o aprendizado progressivo do usuário.

De modo geral, conforme sintetizado na Tabela Y (Etapa 3), a terceira rodada representou um ponto de inflexão no desenvolvimento do Smart Energy Meter, com melhorias significativas em praticamente todos os princípios heurísticos, embora algumas lacunas estruturais — como ausência de comparação contextual e inexistência de mecanismos de desfazer — ainda persistissem e fossem posteriormente retomadas pelos especialistas na etapa final.

A quarta e última etapa da avaliação heurística, conduzida por especialistas externos, teve como objetivo validar as melhorias implementadas ao longo das rodadas anteriores e identificar lacunas remanescentes na interface do Smart Energy Meter. Os resultados indicaram que o aplicativo apresentou avanços substanciais em termos de clareza informacional, previsibilidade e consistência visual. Entretanto, a etapa final também evidenciou pontos críticos que ainda exigem ajustes para consolidar a experiência do usuário.

No que diz respeito à visibilidade do estado do sistema, os especialistas reconheceram que a reestruturação da hierarquia da informação, incluindo a introdução de legendas explicativas, aprimoramento das animações de carregamento e refinamento dos indicadores visuais, tornou a navegação mais fluida e compreensível. As melhorias implementadas proporcionaram ao usuário maior percepção sobre o andamento das ações e sobre a origem dos dados apresentados. Ainda assim, os avaliadores destacaram a persistência de um problema crítico: a ausência de mensagens orientativas quando não havia dispositivos cadastrados ou sensores conectados. A tela vazia, sem explicações, comprometia a previsibilidade do sistema e poderia induzir o usuário a acreditar que o aplicativo estava inoperante. Em resposta a esse achado, foi inserida a mensagem “Nenhum dispositivo cadastrado. Adicione um dispositivo para visualizar os dados”, alinhando-se ao princípio de \cite{Nielsen1994} sobre feedback contínuo e informativo.

A avaliação final também revelou questões importantes relativas à consistência e padrões. Embora grande parte da interface já apresentasse padronização visual melhorada, os especialistas observaram que alguns elementos ainda careciam de maior uniformidade, como ícones, espaçamentos e contrastes, particularmente na tela de login e nas áreas de cadastro de dispositivos. Essas inconsistências foram associadas à ausência de um sistema de design unificado, cuja implementação foi sugerida para garantir robustez visual e previsibilidade funcional. A tela de QR Code foi identificada como o ponto mais crítico nesse aspecto: em alguns dispositivos, apresentou travamentos durante o carregamento, interrompendo o fluxo de cadastro e comprometendo a confiança do usuário na aplicação. O problema se tornou ainda mais evidente pela ausência de mensagens informativas que explicassem o motivo do erro ou orientassem sobre como prosseguir.

Quanto à recuperação e prevenção de erros, os especialistas elogiaram a evolução das mensagens de erro, agora mais descritivas e direcionadas, atendendo parcialmente às recomendações de \cite{Norman2002}. No entanto, ressaltaram que o aplicativo ainda não comunicava adequadamente cenários de falha relacionados à ausência de sensores conectados, tampouco aos erros de leitura do QR Code. A falta de mensagens claras nesses contextos poderia resultar em interpretações equivocadas por usuários iniciantes, aumentando a probabilidade de abandono da tarefa.

No âmbito da navegação, controle e liberdade do usuário, a etapa final apontou melhorias significativas, como a inclusão de botões de retorno e a reorganização de menus. Contudo, avaliadores relataram dificuldade no acesso ao menu lateral, cujo ícone de perfil no canto superior esquerdo era pequeno e pouco responsivo, configurando uma barreira à descoberta de funcionalidades. A ausência de atalhos diretos para configurações também foi citada como uma limitação, sugerindo que futuras versões incorporem pontos de acesso rápido às ações mais frequentes.

A análise referente à estética e design minimalista indicou que, embora o aplicativo apresentasse uma interface limpa e consistente com o conceito de simplicidade, ainda carecia de reforços na hierarquia visual. Em especial, faltavam diferenciações claras entre ações primárias e secundárias e maior ênfase em elementos essenciais para orientar o usuário. Os especialistas consideraram que o minimalismo aplicado era funcional, mas que poderia beneficiar-se de refinamentos visuais que tornassem a navegação mais acessível e intuitiva, sobretudo para usuários sem familiaridade com monitoramento energético.

Por fim, no que se refere à documentação e suporte, os especialistas confirmaram que a seção de ajuda estava mais acessível do que nas etapas anteriores, mas ainda apresentava desafios de organização e profundidade. A ausência de conteúdo multimídia (como vídeos explicativos, diagramas interativos e exemplos práticos) foi apontada como uma limitação. Além disso, avaliadores relataram dificuldade em encontrar opções claras de suporte técnico ou contato direto, o que é especialmente sensível em aplicativos que envolvem dispositivos IoT. A literatura (\cite{Dix2004}) reforça que suporte acessível e contextual é indispensável para reduzir a frustração do usuário, aspecto que permanece como recomendação para versões posteriores.

Em síntese, conforme sistematizado na Tabela Z (Etapa 4), a última rodada de avaliação validou avanços importantes, especialmente em clareza informacional, feedback e previsibilidade, mas evidenciou problemas críticos relacionados à ausência de mensagens em cenários de não conexão, inconsistências visuais pontuais e falhas no fluxo de QR Code. Os especialistas concluíram que, embora o Smart Energy Meter tenha alcançado um nível elevado de maturidade funcional, ainda há oportunidades de refinamento que podem consolidar a interface como uma solução mais robusta, intuitiva e alinhada às diretrizes heurísticas de Nielsen.

\subsection{Smart Fish Finder}
A primeira etapa da avaliação concentrou-se na inspeção heurística preliminar e na análise exploratória da interface do Smart Fish Finder, realizada por (pesquisador / avaliador) com base em protótipos funcionais e em telas interativas. O objetivo desta etapa foi mapear problemas de usabilidade evidentes em estágios iniciais de desenvolvimento, priorizando questões relativas à visibilidade do estado do sistema, clareza informativa e prevenção de erros. As atividades envolveram a navegação sistemática pelas telas principais (mapa de pesca, painel de sensores, tela de simulação e formulários de configuração), registro de ocorrências em checklist baseado nas heurísticas de Nielsen e observação de fluxos críticos como conexão de sensores e leitura do sonar.

Os resultados desta etapa indicaram que, embora a interface apresente uma organização visual coerente e utilize representações gráficas adequadas (gráficos de profundidade, ícones de status e indicadores de temperatura), existem deficiências que comprometem a interpretabilidade por usuários inexperientes. Em particular, foram identificados: (a) ausência de legendas e rótulos explicativos para símbolos e escalas do sonar; (b) feedbacks de sistema genéricos para eventos críticos (ex.: falha de conexão), sem orientação sobre causas ou ações corretivas; (c) inexistência de indicadores de progresso em processos dependentes de atualização em tempo real, gerando incerteza quanto ao estado das leituras; e (d) espaços vazios e delimitação insuficiente entre blocos informativos, o que reduz a legibilidade e a rápida assimilação de dados. Adicionalmente, verificou-se ausência de elementos de prevenção de erros em fluxos sensíveis (por exemplo, confirmação para exclusão de pontos no mapa), o que aumenta o risco de operações irreversíveis.

Em consonância com a literatura de interação humano-computador, que enfatiza a necessidade de sinalização clara do estado do sistema e orientação contextual ao usuário (\cite{Norman2002, Nielsen1994}), os achados desta etapa motivaram um conjunto inicial de intervenções: inclusão planejada de legendas e tooltips, revisão das mensagens de erro para torná-las diagnósticas e sugestivas de solução, e especificação de indicadores visuais de carregamento para leituras em tempo real. Esses ajustes serviram de base para as iterações subsequentes e para a priorização das melhorias a serem avaliadas na Etapa 3.

A segunda etapa concentrou-se na reavaliação do Smart Fish Finder após a implementação das primeiras correções derivadas da etapa inicial. Essa rodada teve como objetivo verificar a eficácia das melhorias incorporadas, identificar problemas persistentes e avaliar a consistência entre o comportamento esperado do sistema e o comportamento efetivamente observado. A avaliação foi realizada utilizando versões atualizadas do aplicativo em ambiente real de uso, incluindo testes com leituras simuladas e com diferentes cenários de conexão do sensor.

Os resultados dessa fase indicaram avanços significativos na clareza da interface e na organização das informações. Elementos visuais previamente identificados como ambíguos foram parcialmente ajustados, incluindo melhorias na disposição dos ícones, ajustes tipográficos e pequenas reorganizações na área de exibição dos dados do sonar. Apesar disso, verificou-se que algumas dificuldades persistiam, sobretudo relacionadas à ausência de contextualização explícita dos dados exibidos. Usuários sem familiaridade com monitoramento aquático ainda apresentaram dúvidas sobre o significado de variações na profundidade, amplitudes gráficas e indicadores de temperatura.

A reavaliação também identificou avanços modestos no sistema de feedback. Algumas mensagens de erro foram reformuladas para comunicar de forma mais direta falhas de conexão ou a necessidade de pareamento do sensor. Contudo, a maior parte desses feedbacks ainda se mostrou genérica, sem detalhar causas prováveis nem sugerir ações específicas para resolução — o que indica que o sistema permanecia aquém das recomendações para suporte ao usuário e diagnóstico contextualizado, conforme sugerido em heurísticas de prevenção e recuperação de erros (Nielsen, 1993).

No que se refere à visibilidade do estado do sistema, a Etapa 2 confirmou que a interface seguia carecendo de indicadores de progresso em tarefas contínuas, como atualização das leituras do sonar ou carregamento de mapas. Embora algumas animações tenham sido adicionadas em pontos específicos, elas ainda eram pouco perceptíveis ou não estavam presentes em todas as telas críticas, especialmente durante o estabelecimento da conexão com o sensor.
A organização estrutural do aplicativo manteve-se consistente com a lógica concebida, mas a ausência de um fluxo guiado ou introdução inicial continuou sendo um fator limitante, sobretudo para novos usuários. A navegação, apesar de funcional, ainda apresentava pontos de incerteza, como telas sem botão de retorno explícito ou ações cujos efeitos não eram devidamente explicados ao usuário.

De modo geral, a Etapa 2 demonstrou que o Smart Fish Finder avançou em direção a uma interface mais clara e estruturada, mas ainda apresentava limitações relativas à legibilidade dos dados, à completude do feedback e à consistência na sinalização de estados do sistema. Os achados dessa rodada foram essenciais para orientar o conjunto de melhorias estruturais e informativas que seriam implementadas posteriormente e avaliadas na Etapa 3.

\begin{figure}[!ht]
    \begin{center}
        \includegraphics[width=\columnwidth]{figures/resultado/Imagem4.png}
        \caption{Evolução da Ajuda e Documentação – Smart Energy Meter}
        \label{Fig4}
    \end{center}
\end{figure}

A terceira etapa teve como foco a validação das melhorias implementadas a partir das análises anteriores e a identificação de ajustes adicionais necessários para consolidar a usabilidade do Smart Fish Finder. Diferentemente das etapas iniciais, essa rodada avaliativa utilizou versões mais maduras do aplicativo, já contendo correções estruturais, revisões de layout e aprimoramentos funcionais. O objetivo foi verificar se as alterações efetivamente reduziram as dificuldades encontradas pelos usuários e se os elementos críticos de interação estavam mais claros e consistentes.

Os resultados dessa fase indicaram avanços significativos na clareza informativa do sistema. A interface passou a apresentar melhor organização visual, incluindo ajustes na distribuição dos ícones, maior separação entre seções e uso mais evidente de divisores gráficos. Também foram adicionadas animações para sinalizar alterações na profundidade e movimentação dos peixes, aumentando a visibilidade do estado do sistema e reforçando a percepção de atualização contínua dos dados.

Outro aspecto aprimorado nesta etapa foi a comunicação de erros e estados excepcionais. A partir das recomendações da Etapa 2, foram incluídas mensagens mais explicativas para situações como falhas na conexão Bluetooth, ausência de sensores configurados ou interrupções na leitura dos dados. Essas mensagens passaram a apresentar não apenas o erro, mas também orientações para solução, alinhando-se às heurísticas de prevenção e recuperação de erros (\cite{Norman2002, Nielsen1994}). Além disso, foram implementados alertas visuais e notificações pop-up que tornaram o reconhecimento de falhas mais perceptível ao usuário.

No que se refere à terminologia, observou-se uma evolução positiva. Termos frequentemente utilizados pelos pescadores, como “profundidade”, “qualidade da água” e “movimentação dos peixes”, foram mantidos, porém complementados por explicações adicionais e dicas contextuais. Esse refinamento buscou reduzir ambiguidades, especialmente para usuários iniciantes que ainda não dominavam conceitos específicos da pesca assistida por sensores.

Apesar dos avanços, algumas limitações persistiram. A ausência de uma introdução guiada ou tutorial inicial continuou dificultando o primeiro contato dos usuários com o aplicativo. Da mesma forma, verificou-se a necessidade de ampliar as opções de personalização, já que diferentes perfis de pescadores demonstraram preferências distintas quanto à forma de visualizar dados de profundidade, temperatura e qualidade da água. Adicionalmente, algumas telas ainda apresentavam pouca orientação para ações de navegação, o que motivou a inclusão de gestos adicionais e botões de retorno mais acessíveis.

A Etapa 3 também evidenciou melhorias importantes nos fluxos críticos do sistema, como confirmação de ações potencialmente irreversíveis (ex.: exclusão de dados ou redefinição de parâmetros). Foram adicionados diálogos de confirmação e aprimoradas as respostas de erro no backend, permitindo diagnósticos mais confiáveis e reduzindo riscos de perda acidental de informações.

De forma geral, a terceira etapa consolidou as principais correções estruturais e informativas do Smart Fish Finder, resultando em uma interface mais robusta, clara e alinhada às necessidades dos usuários, embora ainda com oportunidades de aprimoramento relacionadas à orientação inicial e à personalização da apresentação dos dados. Essas análises finais forneceram subsídios fundamentais para a etapa de avaliação especializada e para recomendações futuras de desenvolvimento.

A quarta etapa consistiu na avaliação do Smart Fish Finder por especialistas em usabilidade, cujo objetivo foi validar as melhorias implementadas nas etapas anteriores e identificar pontos de refinamento ainda necessários. Essa rodada revelou que, embora o aplicativo apresentasse avanços substanciais em termos de organização visual, feedback de sistema e coerência operacional, persistiam limitações relacionadas à clareza informativa, orientação inicial do usuário e consistência nos elementos gráficos.

Os especialistas destacaram, inicialmente, a ausência de um onboarding estruturado, recomendando a implementação de uma tela de boas-vindas interativa que apresentasse os principais recursos do aplicativo e contextualizasse o funcionamento do sensor. Essa lacuna impactava principalmente usuários iniciantes, que tinham dificuldades em compreender como proceder quando acessavam o aplicativo sem um sensor conectado. Para mitigar esse problema, sugeriu-se a inclusão de mensagens específicas indicando a necessidade de pareamento, além de notificações que orientassem o usuário quando não houvesse dados disponíveis ou quando o sonar não detectasse atividade.

Outro ponto amplamente discutido na etapa especializada foi a insuficiência de legendas e explicações visuais para cores, símbolos e métricas do sonar. Os avaliadores enfatizaram que elementos essenciais, como o significado das colorações (verde, amarelo, vermelho) e os indicadores de profundidade, necessitavam de uma legenda explícita. Recomendaram, ainda, a inserção de ícones descritivos para facilitar a interpretação de parâmetros como temperatura e qualidade da água. Esses ajustes foram considerados fundamentais para reduzir ambiguidade e favorecer a interpretação dos dados.

\begin{figure}[!ht]
    \begin{center}
        \includegraphics[width=\columnwidth]{figures/resultado/Imagem5.png}
        \caption{Evolução da Ajuda e Documentação – Smart Energy Meter}
        \label{Fig5}
    \end{center}
\end{figure}

Em relação ao fluxo de navegação, os especialistas reconheceram melhorias implementadas anteriormente, como botões de retorno e diálogos de confirmação para ações críticas. Contudo, sugeriram a adoção adicional de gestos de deslizar para facilitar transições entre telas, sobretudo em dispositivos móveis. Também recomendaram ajustes em ícones pouco intuitivos. O ícone de “menu hambúrguer”, por exemplo, utilizado na tela de sensores, induzia à interpretação equivocada de um menu geral. Assim, propôs-se sua substituição por um ícone de engrenagem, em conformidade com convenções amplamente aceitas de design (\cite{Shneiderman2016}).

Outro aspecto identificado foi a necessidade de aprimorar a visualização do processamento dos dados coletados pelo sensor. Os especialistas relataram que a ausência de indicadores progressivos dificultava a compreensão do usuário sobre o estado de carregamento ou atualização dos dados. Sugeriram, portanto, a implementação de barras de progresso, animações mais evidentes e, opcionalmente, feedback sonoro para indicar eventos relevantes, como detecção de atividade subaquática.

A avaliação também apontou limitações na personalização da interface. Embora algumas opções tenham sido adicionadas, como ajustes manuais de profundidade e controle da velocidade de rolagem, os avaliadores recomendaram a inclusão de um “modo avançado”. Esse modo permitiria configurações mais detalhadas, atendendo usuários experientes que desejam ajustar escalas, unidades de medida ou a disposição dos dados no visor principal.

Quanto ao design visual, os especialistas reconheceram progressos, especialmente no refinamento dos contrastes e no espaçamento dos elementos. Entretanto, identificaram necessidade de maior eficiência na organização da informação, evitando rolagem excessiva e garantindo que os principais indicadores estejam sempre visíveis. Também recomendaram aprimorar as cores e destaques utilizados nos avisos de erro, complementando-as com códigos específicos que facilitem a rápida identificação do tipo de problema.

Por fim, a análise especializada reiterou a importância de uma documentação mais acessível. Embora a seção de ajuda estivesse disponível, os avaliadores observaram que sua organização poderia ser mais intuitiva e que seria benéfica a inclusão de conteúdos multimídia, como vídeos explicativos e infográficos, para facilitar o aprendizado.

Assim, a Etapa 4 consolidou a avaliação final do Smart Fish Finder, confirmando que o aplicativo apresentou avanços significativos em usabilidade, mas ainda requer refinamentos específicos na comunicação de estados do sistema, na orientação inicial do usuário, na padronização visual e na personalização avançada da interface. Essas recomendações constituem a base para a evolução futura do sistema, garantindo maior acessibilidade, consistência e eficiência no uso.

\subsection{Smart Flow Hall}

A primeira etapa concentrou-se na análise inicial da interface do Smart Flow Hall, utilizando protótipos em Figma e versões preliminares disponibilizadas via TestFlight. Nessa fase, observou-se que o aplicativo apresentava uma estrutura funcional básica, permitindo navegação pelas principais funcionalidades, como visualização dos sensores instalados, leitura de vazão e monitoramento do histórico. Contudo, a interface ainda demonstrava problemas significativos relacionados à clareza das informações e à falta de elementos orientadores para o usuário. A navegação era pouco intuitiva, demandando memorização excessiva das etapas necessárias para acessar determinadas funções.

As mensagens de erro eram genéricas, dificultando a compreensão das causas de falhas, especialmente em cenários envolvendo comunicação Bluetooth ou ausência de dados do sensor. Além disso, verificou-se a ausência de mecanismos adequados de feedback, como indicadores de carregamento ou confirmação de ações, o que aumentava a sensação de incerteza durante o uso. A terminologia adotada também se mostrava excessivamente técnica para parte do público-alvo, o que poderia gerar barreiras de entendimento. De modo geral, a Etapa 1 revelou um aplicativo funcional, porém com limitações que impactavam diretamente a usabilidade e a interpretabilidade das informações apresentadas.

Na segunda etapa, o aplicativo foi avaliado em versões experimentais disponibilizadas nas lojas Google Play e Apple App Store, permitindo uma análise em ambiente mais próximo do uso real. Essa rodada evidenciou avanços importantes em relação à primeira versão, sobretudo na estabilidade do sistema e na apresentação dos dados coletados pelos sensores.

A comunicação com os dispositivos físicos mostrou-se mais consistente, e algumas telas foram reorganizadas para tornar a navegação menos linear. Apesar dessas melhorias, persistiam lacunas relevantes, especialmente na forma como o aplicativo orientava o usuário em processos críticos, como configuração inicial dos sensores e interpretação dos valores de vazão. Embora as mensagens de erro e notificações tivessem sido ampliadas, ainda careciam de clareza para auxiliar o diagnóstico de problemas. 

A ausência de legendas para gráficos e indicadores também permaneceu, dificultando a compreensão do comportamento da vazão em diferentes períodos. Do ponto de vista visual, o layout apresentava melhor equilíbrio, mas ainda havia excesso de áreas vazias e pouca hierarquia entre os elementos informativos. A Etapa 2, portanto, consolidou avanços técnicos, mas indicou a necessidade de aprimoramentos adicionais na comunicação e na orientação ao usuário.

A terceira etapa consistiu na validação e revisão das melhorias implementadas a partir dos achados anteriores. Nessa fase, observou-se que os desenvolvedores haviam incorporado elementos importantes, como ajustes de layout, reorganização dos indicadores de vazão e introdução de feedbacks mais detalhados para falhas de conexão. 

A navegação tornou-se mais coerente, e o fluxo entre as telas ganhou consistência, reduzindo ambiguidades e facilitando a compreensão das etapas de uso. A interface também passou a incluir mensagens explicativas em pontos críticos, sobretudo durante a leitura dos dados do sensor e na interpretação das variações de vazão. Ainda assim, algumas limitações permaneceram, como a ausência de guias introdutórios para novos usuários e a falta de personalização na exibição dos dados. 

A organização dos gráficos e históricos também passou por melhorias, mas os especialistas indicaram que a hierarquização visual poderia ser ainda mais refinada. Em síntese, a Etapa 3 representou um avanço expressivo na maturidade da interface, mas apontou oportunidades adicionais para elevar a clareza e a eficiência da comunicação com o usuário.

A quarta etapa reuniu a análise crítica de dois especialistas externos, utilizando checklist semiestruturado baseado nas heurísticas de Nielsen. Essa rodada confirmou os avanços implementados nas etapas anteriores e destacou a solidez crescente do aplicativo, especialmente no que se refere à consistência visual, precisão dos dados apresentados e estabilidade da comunicação com o sensor. 

\begin{figure}[!ht]
    \begin{center}
        \includegraphics[width=\columnwidth]{figures/resultado/Imagem6.png}
        \caption{Evolução da Ajuda e Documentação – Smart Energy Meter}
        \label{Fig5}
    \end{center}
\end{figure}

No entanto, os avaliadores identificaram aspectos que ainda limitavam a experiência de uso, principalmente relacionados à ausência de elementos de orientação inicial e de explicações diretas sobre os indicadores de vazão. A falta de legendas para gráficos, a necessidade de indicadores de progresso mais evidentes e a ausência de feedback contextualizado para situações sem leitura de dados foram pontos críticos ressaltados pelos especialistas. 

Além disso, recomendou-se a substituição de alguns ícones por alternativas mais universalmente reconhecíveis, bem como a inclusão de mensagens informando a necessidade de conectar um sensor antes de iniciar o monitoramento. A avaliação também sugeriu ampliar as opções de personalização e reorganizar o layout para que informações prioritárias fossem acessadas com maior rapidez. Assim, a Etapa 4 consolidou a revisão final da interface, validou melhorias implementadas e apresentou recomendações fundamentais para a versão final do aplicativo.

\subsection{Síntese dos Resultados}
Os resultados obtidos nas quatro rodadas de avaliação evidenciam que, embora os três aplicativos analisados — Smart Energy Meter, Smart Fish Finder e Smart Flow Hall — pertençam a domínios distintos, eles compartilham desafios estruturais que são característicos de interfaces voltadas a sistemas IoT. Em suas versões iniciais, todos apresentaram limitações relacionadas à visibilidade do estado do sistema, clareza informacional, prevenção de erros e orientação ao usuário, reforçando achados já discutidos na literatura sobre a importância de reduzir a complexidade cognitiva em ambientes onde o usuário precisa interpretar dados em tempo real (\cite{Norman2002, Nielsen1994}).

A análise comparativa permitiu identificar que sistemas IoT demandam um nível elevado de transparência operacional, especialmente no que se refere ao comportamento dos sensores, conectividade e atualização contínua das leituras. Nos três aplicativos, a ausência inicial de indicadores de progresso e mensagens de contexto gerou incertezas sobre o funcionamento do sistema, reforçando que a visibilidade do estado é um elemento crítico para esse tipo de aplicação. Similarmente, questões relacionadas à terminologia e ao significado dos ícones revelaram a necessidade de mediação interpretativa, aspecto fundamental para tornar os dados acessíveis a usuários com diferentes níveis de expertise, conforme sugerem \cite{Shneiderman2016}.

Outro achado consiste na constatação de que a aplicação sistemática das heurísticas ao longo de múltiplas etapas resultou em melhorias consistentes, tanto no plano visual quanto funcional. A evolução observada ao longo das quatro rodadas demonstrou que ciclos iterativos favorecem a maturação do design, permitindo ajustes sucessivos em elementos como hierarquia da informação, mensagens de erro, fluxos de navegação e organização dos dados. Assim, o estudo oferece uma contribuição metodológica relevante ao mostrar que a avaliação heurística em múltiplas fases pode servir como um modelo replicável para o desenvolvimento de interfaces IoT, principalmente em contextos urbanos, onde a automação depende de experiências fluídas, intuitivas e previsíveis.

Do ponto de vista prático, os achados indicam que melhorias como personalização da interface, tutoriais interativos, indicadores refinados de carregamento e mensagens de contexto podem aumentar substancialmente a eficiência e acessibilidade desses aplicativos. Tais soluções, alinhadas a princípios de design responsivo e de acessibilidade, tornam os sistemas mais democráticos e adaptáveis a diferentes perfis de usuários (\cite{Dix2004}). Além disso, as avaliações revelaram que mecanismos de prevenção de erros e confirmação de ações críticas são essenciais para garantir segurança na interação, especialmente em ambientes onde a confiabilidade do sistema está diretamente associada ao uso de sensores e coleta de dados ambientais.

O conjunto dessas evidências sugere que o presente estudo contribui não apenas para o aprimoramento dos aplicativos avaliados, mas também para o avanço do conhecimento sobre design de interação para sistemas IoT. Ao demonstrar como problemas recorrentes podem ser mitigados por meio de ciclos iterativos e triangulação entre avaliadores, o estudo reforça a importância de metodologias de avaliação contínua em ecossistemas de automação e monitoramento.

\section{Conclusion}
Este estudo analisou a usabilidade de aplicativos de controle de dispositivos IoT voltados à automação, integrando diferentes metodologias para avaliar a experiência do usuário e a aderência dessas interfaces aos princípios de design centrado no usuário. A avaliação heurística permitiu identificar padrões recorrentes de uso, desafios específicos e oportunidades de aprimoramento importantes para o contexto das soluções analisadas.

Os resultados indicam que, embora os aplicativos ofereçam funcionalidades relevantes, ainda existem lacunas de usabilidade que podem afetar a adoção e a aceitação dessas tecnologias. Os aplicativos desenvolvidos pela Smart Sensor Design apresentam vantagens relacionadas à flexibilidade e à adaptabilidade às necessidades dos usuários e do ambiente urbano onde se inserem. Assim, as melhorias identificadas neste estudo não apenas são aplicáveis aos aplicativos avaliados, como também fornecem diretrizes para futuros aperfeiçoamentos e novos desenvolvimentos.

A principal contribuição desta pesquisa está na abordagem integrada de avaliação, que possibilitou um diagnóstico aprofundado das interfaces e fundamentou recomendações consistentes em princípios consolidados de usabilidade. Os achados reforçam a importância de um processo iterativo de design, em que interfaces são continuamente ajustadas a partir da análise das interações reais dos usuários e dos padrões de uso observados.

Além das contribuições diretas para os aplicativos analisados, os resultados deste trabalho dialogam com o contexto mais amplo das cidades inteligentes e da engenharia urbana. A expansão do uso de dispositivos IoT em infraestrutura e serviços urbanos demanda soluções que conciliem inovação tecnológica, acessibilidade e eficiência. Nesse sentido, engenheiros urbanos, gestores públicos e tomadores de decisão podem se beneficiar de estudos desse tipo, uma vez que o sucesso de tecnologias inteligentes depende tanto da robustez técnica quanto da qualidade da experiência do usuário.

Em suma, a adoção efetiva desses sistemas exige esforços interdisciplinares que envolvam engenheiros, designers, especialistas em usabilidade e profissionais de tecnologia. A colaboração entre essas áreas pode resultar em interfaces mais consistentes, intuitivas e alinhadas às necessidades dos usuários, fortalecendo o papel da automação na transformação dos ambientes urbanos.

\subsection{Limitações}

Apesar da abordagem integrada e da abrangência dos dados coletados, algumas limitações devem ser reconhecidas. A principal delas refere-se à ausência de testes em ambiente real com sensores conectados e operando continuamente. Embora a quarta rodada da avaliação heurística tenha sido realizada com os aplicativos instalados, a falta de interação com dispositivos reais pode ter influenciado a percepção de certas funcionalidades ou problemas.

Outra limitação diz respeito ao número reduzido de especialistas envolvidos na avaliação. Uma amostra maior e mais diversa poderia ampliar a robustez da análise, oferecendo perspectivas complementares e aprofundando a compreensão dos desafios enfrentados pelos usuários.

\subsection{Trabalhos Futuros}

A partir das recomendações desta pesquisa, espera-se que futuros desenvolvimentos em aplicativos de automação incorporem diretrizes mais robustas de usabilidade, especialmente em relação a:

\begin{enumerate}
    \item \textbf{Refinamento da experiência do usuário}: Interfaces mais claras, previsíveis e orientadas ao controle do usuário, reduzindo barreiras de uso para indivíduos com pouca experiência tecnológica.
    \item \textbf{Flexibilidade na interação}: Inclusão de mecanismos de personalização que permitam adaptar a interface a diferentes perfis e contextos de uso.
    \item \textbf{Aprimoramento do feedback e comunicação visual}: Criação de respostas visuais e informacionais mais eficazes, reduzindo a carga cognitiva e facilitando a compreensão das ações do sistema.
    \item \textbf{Integração de metodologias complementares}: Adoção de avaliações contínuas e científicas, garantindo ciclos de validação e melhoria constantes.
\end{enumerate}

Este último aspecto torna-se ainda mais relevante diante de trabalhos já em andamento. A continuidade das avaliações heurísticas e práticas pode aumentar a precisão das melhorias implementadas e apoiar a evolução dos aplicativos da Smart Sensor Design. A incorporação de outros aplicativos ao escopo da análise também pode fortalecer a maturidade da empresa em termos de design e usabilidade.

Como perspectivas futuras, sugere-se a investigação de soluções baseadas em inteligência artificial para adaptação dinâmica das interfaces, bem como o desenvolvimento de estudos longitudinais que acompanhem a evolução da experiência do usuário em cenários reais de automação. Avaliações complementares também podem explorar o impacto de novas metodologias científicas de análise de interfaces, contribuindo para um ciclo sistemático de aperfeiçoamento contínuo.

Dessa forma, esta pesquisa evidencia não apenas os desafios da usabilidade em sistemas de automação, mas também apresenta um caminho estruturado para o aprimoramento das interfaces analisadas, promovendo maior aderência tecnológica, eficiência operacional e uma experiência mais fluida e intuitiva para os usuários.



% \begin{table*}[!ht]
% \centering
% \footnotesize
% \begin{tabular}{cp{3.8cm} p{3.8cm} p{3.8cm} p{4.2cm}}
% \hline
% \textbf{Nº} &
% \textbf{Rodada 1} &
% \textbf{Rodada 2} &
% \textbf{Rodada 3} &
% \textbf{Principais Evoluções} \\
% \hline
%  & & & & \\

% 1 &
% Estado pouco visível; ausência de indicadores. &
% Status básico informado, mas sem contexto. &
% Sistema interpretativo completo; indicadores claros. &
% Progressiva contextualização e clareza dos dados apresentados. \\
%  & & & & \\

% 2 &
% Termos técnicos sem explicação. &
% Tooltips, cores semafóricas e ícones adicionados. &
% Comparações práticas e metáforas visuais reforçadas. &
% Aumento contínuo de compreensão e aproximação ao mundo real. \\
%  & & & & \\

% 3 &
% Ações irreversíveis sem confirmação. &
% Confirmações básicas; ausência de desfazer. &
% Confirmações ampliadas e alternativas sugeridas. &
% Ações tornam-se mais seguras e guiadas. \\
%  & & & & \\

% 4 &
% Padrões visuais básicos presentes. &
% Padronização parcial de cores e ícones. &
% Design system totalmente unificado. &
% Identidade visual progressivamente unificada. \\
%  & & & & \\

% 5 &
% Validação insuficiente. &
% Alertas e validações básicas. &
% Validação em tempo real e feedback dinâmico. &
% Evolução para um sistema robusto de prevenção de erros. \\
%  & & & & \\

% 6 &
% Alta carga de memorização. &
% Hierarquia visual melhorada. &
% Navegação intuitiva e atalhos úteis. &
% Redução significativa da carga cognitiva. \\
%  & & & & \\

% 7 &
% Navegação linear rígida. &
% Redução de etapas em alguns fluxos. &
% Fluxos otimizados e menus reorganizados. &
% Aumento contínuo de eficiência e flexibilidade. \\
%  & & & & \\

% 8 &
% Minimalismo excessivo e desorientador. &
% Design mais equilibrado. &
% Hierarquia visual clara e responsividade. &
% Melhor equilíbrio entre simplicidade e orientação. \\
%  & & & & \\

% 9 &
% Mensagens genéricas. &
% Mensagens mais específicas, porém limitadas. &
% Classificação por tipo e instruções corretivas. &
% Feedback evolui para contextual e orientador. \\
%  & & & & \\

% 10 &
% Ajuda técnica e pouco acessível. &
% Textos revisados e linguagem simplificada. &
% Ajuda contextual e links relevantes. &
% Melhoria contínua na acessibilidade e suporte ao usuário. \\
% \hline
% \end{tabular}
% \caption{Evolução dos problemas de usabilidade ao longo das três rodadas segundo os Princípios de Nielsen (numerados de 1 a 10).}
% \end{table*}
\begin{declarations}

\begin{acknowledgements}
Os autores agradecem à equipe técnica da Smart Sensor Design (SSD) pelo suporte oferecido durante o processo de avaliação dos aplicativos, especialmente pela disponibilização dos protótipos, versões beta e acesso restrito às plataformas de teste. Agradecemos também aos especialistas convidados que contribuíram com a revisão técnica na quarta rodada de análise. 
\end{acknowledgements}

\begin{funding}
Este estudo não recebeu financiamento específico de agências de fomento públicas, comerciais ou sem fins lucrativos. A pesquisa foi conduzida com apoio institucional da Smart Sensor Design (SSD), responsável pela disponibilização dos materiais e aplicações avaliadas.
\end{funding}

\begin{contributions}
Conforme a Taxonomia CRediT, as contribuições dos autores foram as seguintes:  
\textbf{Conceituação}: autor principal;  
\textbf{Metodologia}: autor principal;  
\textbf{Coleta de dados e execução das rodadas de avaliação}: autor principal;  
\textbf{Análise dos dados}: autor principal;  
\textbf{Revisão técnica}: especialistas convidados na quarta rodada;  
\textbf{Redação – rascunho original}: autor principal;  
\textbf{Redação – revisão e edição}: todos os autores contribuíram com revisões.  
Todos os autores leram e aprovaram a versão final do manuscrito.
\end{contributions}

\begin{interests}
Os autores declaram que não possuem conflitos de interesse relacionados a esta pesquisa. Apesar do apoio institucional para acesso aos aplicativos, a Smart Sensor Design (SSD) não influenciou o desenho metodológico, a análise ou a interpretação dos resultados.
\end{interests}

\begin{materials}
Os materiais utilizados neste estudo, incluindo o checklist heurístico, a lista de problemas identificados e os protótipos utilizados para as rodadas de avaliação, estão disponíveis no repositório público: \textit{(insira link após disponibilização, como Zenodo, OSF ou GitHub)}. As versões dos aplicativos pertencem à Smart Sensor Design e foram utilizadas exclusivamente para fins de pesquisa, não podendo ser redistribuídas.
\end{materials}

\begin{furtherinformation}
Este artigo fez uso de ferramentas de IA generativa exclusivamente para revisão textual e aprimoramento da clareza narrativa, não afetando o conteúdo metodológico, técnico ou analítico do estudo.

\textbf{Citation Diversity Statement}: Os autores reconhecem a importância da diversidade de citações e se comprometeram a buscar equilíbrio de gênero, nacionalidade e representatividade nas referências utilizadas, conforme sugerido por práticas contemporâneas de equidade acadêmica.
\end{furtherinformation}

\end{declarations}


%Pay attention to the examples in the refs.bib file
%Full DOI address is mandatory whenever available
%Updated last access date (i.e., later than the paper acceptance notification) is required for urls in footnotes and references (except DOI)

\bibliographystyle{apalike-sol}
\balance
\bibliography{refs}

\end{document}

